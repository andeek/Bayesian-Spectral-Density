\documentclass{article}\usepackage[]{graphicx}\usepackage[]{color}
%% maxwidth is the original width if it is less than linewidth
%% otherwise use linewidth (to make sure the graphics do not exceed the margin)
\makeatletter
\def\maxwidth{ %
  \ifdim\Gin@nat@width>\linewidth
    \linewidth
  \else
    \Gin@nat@width
  \fi
}
\makeatother

\definecolor{fgcolor}{rgb}{0.345, 0.345, 0.345}
\newcommand{\hlnum}[1]{\textcolor[rgb]{0.686,0.059,0.569}{#1}}%
\newcommand{\hlstr}[1]{\textcolor[rgb]{0.192,0.494,0.8}{#1}}%
\newcommand{\hlcom}[1]{\textcolor[rgb]{0.678,0.584,0.686}{\textit{#1}}}%
\newcommand{\hlopt}[1]{\textcolor[rgb]{0,0,0}{#1}}%
\newcommand{\hlstd}[1]{\textcolor[rgb]{0.345,0.345,0.345}{#1}}%
\newcommand{\hlkwa}[1]{\textcolor[rgb]{0.161,0.373,0.58}{\textbf{#1}}}%
\newcommand{\hlkwb}[1]{\textcolor[rgb]{0.69,0.353,0.396}{#1}}%
\newcommand{\hlkwc}[1]{\textcolor[rgb]{0.333,0.667,0.333}{#1}}%
\newcommand{\hlkwd}[1]{\textcolor[rgb]{0.737,0.353,0.396}{\textbf{#1}}}%

\usepackage{framed}
\makeatletter
\newenvironment{kframe}{%
 \def\at@end@of@kframe{}%
 \ifinner\ifhmode%
  \def\at@end@of@kframe{\end{minipage}}%
  \begin{minipage}{\columnwidth}%
 \fi\fi%
 \def\FrameCommand##1{\hskip\@totalleftmargin \hskip-\fboxsep
 \colorbox{shadecolor}{##1}\hskip-\fboxsep
     % There is no \\@totalrightmargin, so:
     \hskip-\linewidth \hskip-\@totalleftmargin \hskip\columnwidth}%
 \MakeFramed {\advance\hsize-\width
   \@totalleftmargin\z@ \linewidth\hsize
   \@setminipage}}%
 {\par\unskip\endMakeFramed%
 \at@end@of@kframe}
\makeatother

\definecolor{shadecolor}{rgb}{.97, .97, .97}
\definecolor{messagecolor}{rgb}{0, 0, 0}
\definecolor{warningcolor}{rgb}{1, 0, 1}
\definecolor{errorcolor}{rgb}{1, 0, 0}
\newenvironment{knitrout}{}{} % an empty environment to be redefined in TeX

\usepackage{alltt}
\usepackage{graphicx, hyperref, soul}
\usepackage{amssymb,amsmath,amsthm} 
\usepackage[margin=1.25in]{geometry}


\usepackage[backend=bibtex, natbib=true]{biblatex}
\addbibresource{../references/references.bib}

\usepackage{color}
\newcommand{\ak}[1]{{\color{magenta} #1}}
\newcommand{\mj}[1]{{\color{blue} #1}}

\theoremstyle{plain}
\newtheorem*{res}{Result}

\title{Bayesian Estimation of the Spectral Density \\ {\small Project Proposal}}
\author{Andee Kaplan \& Maggie Johnson}
\IfFileExists{upquote.sty}{\usepackage{upquote}}{}

\begin{document}

\maketitle

\section*{Main Goal}
To estimate the spectral density of a stationary time series using Bayesian methods.

\section*{Details}
\begin{res}
Let $f(\omega_r) \not= 0, 1 \le r \le k$ where $f$ denotes the spectral density of a stationary time series $\{X_t\}$. Then when $n\rightarrow \infty$ the joint distribution of the periodogram at $\omega_r$, $I_n(\omega_r)$, tends to that of $k$ mutually indepependent random variables distributed as $\text{Exponential}(2\pi f(\omega_r))$ for $0<\omega_r<\pi$ \cite{JAZ:4921656}.
\end{res}

We would like to use the asymptotic distributional properties of a periodogram to obtain estimates of the spectral density at $\omega_r$.

\begin{align*}
I_n(\omega_r) |f(\omega_r) &\stackrel{\cdot}{\sim}\text{ indep } \text{Exp}(2\pi f(\omega_r)) \\
f(\omega_r) & \sim \pi(\theta)\\
f(\omega_r) | I_n(\omega_r) &\propto 2\pi f(\omega_r) e^{-2\pi f(\omega_r) y} \pi(\theta)
\end{align*}

In order to implement this method, two fundamental sets of questions need to be explored. These involve
\begin{enumerate}
\item Prior distributions on $\omega_r$
\item The asymptotic property relies on fixed frequencies.
\end{enumerate}


\subsection*{Prior Distributions}
A natural choice of prior for this analysis would be a conjugate Gamma distribution. We would also like to explore the possibility of using an improper prior.

\subsection*{Fixed Frequencies}
It is common practice when dealing with the periodogram to work with the Fourier frequencies $\mathcal{F}_n$ of which there are $n$. However, the asymptotic result holds for a fixed number of frequencies $k$. We would like to explore at what ratio $\frac{k}{n}$ do the approximate independence and exponential distribution assumptions in the result break down in the periodogram values.

The goal would be to develop a procedure to find the largest acceptable $k$ that still maintains the asymptotic result. This procedure would be completed before the Bayesian estimation of the spectral density.

In order to complete this step, simulations from known spectral densities will be used to estimate periodograms created from evenly spaced frequencies partitioning $(0,\pi)$ in decreasing compactness starting at $n$ (Fourier frequencies). Monte Carlo approximations of first and second moments and a $\chi^2$ goodness-of-fit test will be used to assess if the asymptotic result holds. 

\section*{Application}
We will use our Bayesian estimation method of the spectral density on a real dataset of remotely sensed phenology data from India. The data consists of 2500 locations each with a time-series of weekly measurements of chorophyll content over five years. A sample of locations will be used to implement the method.

\printbibliography
\end{document}
